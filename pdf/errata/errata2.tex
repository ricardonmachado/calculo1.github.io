\documentclass[12pt,a4paper]{report}
\usepackage[utf8]{inputenc}
\usepackage{indentfirst}
\usepackage{amssymb,amsmath,amsthm,amsfonts,amscd}	%s�mbolos e caracteres especiais
\usepackage{empheq}
\usepackage{latexsym}

\usepackage{graphicx,graphics,epsfig}								%para inserir figuras
\usepackage{graphpap}										%pictures
\usepackage{float}
\usepackage{xcolor}
\usepackage[brazil]{babel}

\usepackage[T1]{fontenc}
\usepackage{graphicx}
\usepackage[normalem]{ulem}
\usepackage[a4paper,top=20mm,bottom=20mm,left=20mm,right=20mm]{geometry}
\usepackage{textcomp}
\usepackage{amsthm}
\usepackage[normalem]{ulem}
\usepackage{multicol}


\newtheorem{q}{}


\usepackage{array}



\usepackage{tcolorbox}% cria caixas para inserir texto
\tcbset{boxrule=0.2mm,colback=white} % configuração da caixa tcolorbox
\usepackage{verbatim}
\usepackage{enumitem} % Customize lists
%\usepackage{index}
% \usepackage{splitidx}
\usepackage{listings}
\lstset{
    escapeinside={(*}{*)}
}
%\usepackage[scale=0.85]{beramono}
% \usepackage{imakeidx}
%\usepackage[scale=10pt]{FiraMono}
%\usepackage[scale=0.85]{DejaVuSansMono}

%\usepackage[scale=0.85]{sourcecodepro}

\usepackage[scaled=0.80]{DejaVuSansMono}
%\usepackage[scale=0.85]{inconsolata}
%\usepackage[scale=0.82]{GoMono}
\newcommand{\un}{\underline{ }}
% \newcommand{\idx}[1]{\index[metodo]{{\tt #1}}}


\definecolor{mygray}{rgb}{0.42,0.42,0.42}
\usepackage{upquote}

\lstset{
  language=Python,
  showstringspaces=false,
  numberstyle= \scriptsize,
  numbers=left,
  xleftmargin=1em,
  xrightmargin=0.4em,
  numbersep=5pt,
  formfeed=\newpage,
  tabsize=4,
  commentstyle=\color{mygray}\itshape,
  %basicstyle=\footnotesize\fontfamily{fvm}\selectfont,%\small\ttfamily,
  basicstyle=\ttfamily,
  frame=single,
  literate= {ç}{{\c c}}1  {ú}{{\'u}}1 {í}{{\'i}}1 {â}{{\^a}}1 {ó}{{\'o}}1 {õ}{{\~o}}1 {á}{{\'a}}1 {ã}{{\~a}}1 {é}{{\'e}}1 {É}{{\'E}}1 {ê}{{\^e}}1 {ô}{{\^o}}1 {à}{{\`a}}1,
  morekeywords={srange, lambda}
}


\begin{document}
\pagenumbering{arabic}
\thispagestyle{empty}










\begin{center}
{\bf ERRATA}
\end{center}

\noindent
Silva, L., Santos, M., Machado, R. \textit{Elementos de Computação Matemática com SageMath}. SBM. Rio de Janeiro. 2019.


% folha, linha, onde se lê, leia-se

\vspace{1cm}

\noindent
{\bf Correções nos textos:}
\vspace{1cm}

%\centering
\setlength{\extrarowheight}{1.5pt}
\begin{tabular}{|c|c|c|c|}
\hline
 ~~ Página ~~ & ~~ Linha ~~ & ~~~~~~~~ Onde se lê ~~~~~~~~ & ~~~~~~~~~~~~ Leia-se ~~~~~~~~~~~~ \\
\hline \hline
 72 & 8 & o último é \verb|j-k| & 
 \begin{tabular}
 o último é o maior \\ termo menor que \verb|j|  
 \end{tabular}\\ \hline
 303 & 1 & classe $C^1$  & classe $C^2$ \\ \hline
 & & & \\ \hline
\end{tabular}

\vspace{1cm}

\noindent
{\bf Correções nos códigos:}


\begin{itemize}

 \item Na página 191, adicione na função \verb|complemento_ortogonal_livro| entre as linhas 13 e 14 do código:
\begin{lstlisting}[frame=e, numbers=none, xleftmargin=0em]
temp=[]
for w in base_do_compl:
    temp.append(V(w))
base_do_compl = temp
\end{lstlisting}
\item Na página 215, na última linha, antes de \verb|A.is_similar(B)|, escreva:
\begin{lstlisting}[frame=e, numbers=none, xleftmargin=0em]
sage: A = A.change_ring(QQbar)
sage: B = B.change_ring(QQbar)
\end{lstlisting}

\end{itemize}





\end{document}
